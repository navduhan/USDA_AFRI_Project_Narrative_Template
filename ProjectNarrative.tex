\documentclass{usda-afri-narrative}

%%%%%%%%%%%%%%%%%%%%%%%%%%%%%%%%%%%%%%%%%%%%%%%%%%%%%%%%%%%%%%%%%%%%%%%%%%%%%
% USDA AFRI Project Narrative Template
% 
% This template follows USDA AFRI guidelines for project narratives.
% Fill in each section with your project-specific information.
% Delete these instructional comments before submitting your final document.
%
% CRITICAL REQUIREMENTS:
% - Maximum 18 pages (excluding bibliography)
% - Bibliography must be submitted as a SEPARATE PDF (use Bibliography.tex)
% - 12pt Times New Roman font, 1-inch margins (enforced by this class)
% - Single-spaced or as specified by program announcement
%
% COMPILATION:
% Recommended: lualatex ProjectNarrative.tex (best Times New Roman font)
% Alternative: xelatex or pdflatex
%
% SUBMISSION:
% 1. Compile ProjectNarrative.tex -> Upload as "Project Narrative" (max 18 pages)
% 2. Compile Bibliography.tex -> Upload as "Literature Cited" (separate file)
%%%%%%%%%%%%%%%%%%%%%%%%%%%%%%%%%%%%%%%%%%%%%%%%%%%%%%%%%%%%%%%%%%%%%%%%%%%%%

\begin{document}

%%%%%%%%%%%%%%%%%%%%%%%%%%%%%%%%%%%%%%%%%%%%%%%%%%%%%%%%%%%%%%%%%%%%%%%%%%%%%
% SECTION A: INTRODUCTION
%%%%%%%%%%%%%%%%%%%%%%%%%%%%%%%%%%%%%%%%%%%%%%%%%%%%%%%%%%%%%%%%%%%%%%%%%%%%%
% Purpose: Introduce the problem, provide context, and establish your team's
% qualifications to conduct this research.
%
% What to include:
% - Clear problem statement
% - Relevance to USDA AFRI priorities
% - Overview of your approach (brief)
% - Team expertise and qualifications
%%%%%%%%%%%%%%%%%%%%%%%%%%%%%%%%%%%%%%%%%%%%%%%%%%%%%%%%%%%%%%%%%%%%%%%%%%%%%

\section{Introduction}

\subsection{Background}
% Describe the problem or knowledge gap you're addressing.
% - What is the current state of the field?
% - Why is this problem important for agriculture/food systems?
% - What are the broader implications?
% - Connect to USDA AFRI program priorities
% - Cite relevant literature to support your statements

\subsection{Preliminary Work and Team Expertise}
% Establish credibility for your team to conduct this work.
% - Summarize relevant preliminary data or pilot studies
% - Highlight key publications, patents, or previous grants
% - Describe expertise of each team member
% - Explain why this team is uniquely qualified
% - Mention collaborations and institutional resources

%%%%%%%%%%%%%%%%%%%%%%%%%%%%%%%%%%%%%%%%%%%%%%%%%%%%%%%%%%%%%%%%%%%%%%%%%%%%%
% SECTION B: RATIONALE AND SIGNIFICANCE
%%%%%%%%%%%%%%%%%%%%%%%%%%%%%%%%%%%%%%%%%%%%%%%%%%%%%%%%%%%%%%%%%%%%%%%%%%%%%
% Purpose: Explain WHY this research is needed and HOW it will make a difference.
%
% What to include:
% - Clear justification for the research
% - Knowledge gaps this project will fill
% - Potential impacts on agriculture, food systems, or rural communities
% - Alignment with USDA strategic goals
%%%%%%%%%%%%%%%%%%%%%%%%%%%%%%%%%%%%%%%%%%%%%%%%%%%%%%%%%%%%%%%%%%%%%%%%%%%%%

\section{Rationale and Significance}

\subsection{Rationale}
% Provide the scientific justification for your project.
% - What specific knowledge gap will you address?
% - Why is this the right approach?
% - What makes this research timely and important?
% - How does it build on existing knowledge?
% - What are the scientific challenges to be overcome?

\subsection{Significance}
% Explain the broader impacts and importance of your work.
% - Who will benefit from this research? (farmers, consumers, industry, etc.)
% - What are the potential economic impacts?
% - How will this advance the field?
% - What are the long-term benefits?
% - How does it address national/global challenges?
% - Discuss potential for innovation or transformation

%%%%%%%%%%%%%%%%%%%%%%%%%%%%%%%%%%%%%%%%%%%%%%%%%%%%%%%%%%%%%%%%%%%%%%%%%%%%%
% SECTION C: APPROACH
%%%%%%%%%%%%%%%%%%%%%%%%%%%%%%%%%%%%%%%%%%%%%%%%%%%%%%%%%%%%%%%%%%%%%%%%%%%%%
% Purpose: Describe WHAT you will do and HOW you will do it.
% This is typically the longest section of the narrative.
%
% What to include:
% - Clear, measurable objectives
% - Testable hypotheses
% - Detailed methodology
% - Timeline and milestones
% - Data analysis plans
% - Quality assurance procedures
%%%%%%%%%%%%%%%%%%%%%%%%%%%%%%%%%%%%%%%%%%%%%%%%%%%%%%%%%%%%%%%%%%%%%%%%%%%%%

\section{Approach}

\subsection{Objectives and Hypothesis}
% List clear, specific, and measurable objectives.
% Example format:
% \textbf{Objective 1:} [Action verb] [specific goal] [measurable outcome]
% 
% \textbf{Hypothesis 1:} [Testable prediction based on Objective 1]
%
% Tips:
% - Use numbered objectives (typically 3-5 objectives)
% - Make objectives SMART (Specific, Measurable, Achievable, Relevant, Time-bound)
% - Link each hypothesis to an objective
% - State null and alternative hypotheses when appropriate

\subsection{Methods}
% Provide detailed methodology for each objective.
% Organize by objective or work package.
%
% For each objective, describe:
% - Experimental design (controls, replicates, sample sizes)
% - Materials and procedures (step-by-step when necessary)
% - Data collection methods
% - Statistical analysis approaches
% - Quality control measures
% - Validation procedures
%
% Tips:
% - Use subheadings to organize this section
% - Include enough detail for reviewers to evaluate feasibility
% - Cite methods from literature when appropriate
% - Explain any novel or modified techniques
% - Consider using the \boldline{} command for subsections

\subsection{Stakeholder Involvement}
% Describe how stakeholders will be engaged throughout the project.
% - Who are your key stakeholders? (farmers, industry, extension, etc.)
% - How will they be involved? (advisory board, co-creation, field testing)
% - When will engagement occur? (planning, execution, dissemination)
% - How will you incorporate stakeholder feedback?
% - What mechanisms ensure meaningful participation?

%%%%%%%%%%%%%%%%%%%%%%%%%%%%%%%%%%%%%%%%%%%%%%%%%%%%%%%%%%%%%%%%%%%%%%%%%%%%%
% SECTION D: EXPECTED OUTCOMES AND IMPACTS
%%%%%%%%%%%%%%%%%%%%%%%%%%%%%%%%%%%%%%%%%%%%%%%%%%%%%%%%%%%%%%%%%%%%%%%%%%%%%
% Purpose: Describe the anticipated results and their potential impact.
%
% What to include:
% - Specific, measurable outcomes for each objective
% - Short-term and long-term impacts
% - Scientific, economic, social, and environmental benefits
% - Deliverables (publications, tools, datasets, varieties, practices)
%%%%%%%%%%%%%%%%%%%%%%%%%%%%%%%%%%%%%%%%%%%%%%%%%%%%%%%%%%%%%%%%%%%%%%%%%%%%%

\section{Expected Outcomes and Impacts}
% Describe anticipated results and their broader significance.
%
% Expected Outcomes:
% - What specific results do you expect from each objective?
% - What deliverables will be produced? (data, tools, publications, protocols)
% - What metrics will demonstrate success?
%
% Impacts:
% - Scientific impact: How will this advance the field?
% - Economic impact: Cost savings, increased productivity, market opportunities
% - Social impact: Benefits to farmers, rural communities, consumers
% - Environmental impact: Sustainability, conservation, climate resilience
% - Educational impact: Training, curriculum, outreach materials
%
% Consider organizing by timeline:
% - Near-term impacts (within project period)
% - Medium-term impacts (2-5 years after completion)
% - Long-term impacts (5+ years)

%%%%%%%%%%%%%%%%%%%%%%%%%%%%%%%%%%%%%%%%%%%%%%%%%%%%%%%%%%%%%%%%%%%%%%%%%%%%%
% SECTION E: EXTENSION, EDUCATION, AND COMMUNICATION PLAN
%%%%%%%%%%%%%%%%%%%%%%%%%%%%%%%%%%%%%%%%%%%%%%%%%%%%%%%%%%%%%%%%%%%%%%%%%%%%%
% Purpose: Explain how you will share results with stakeholders and the public.
%
% What to include:
% - Target audiences for each type of communication
% - Specific dissemination activities and products
% - Timeline for outreach activities
% - Metrics to assess outreach effectiveness
%%%%%%%%%%%%%%%%%%%%%%%%%%%%%%%%%%%%%%%%%%%%%%%%%%%%%%%%%%%%%%%%%%%%%%%%%%%%%

\section{Extension, Education, and Communication Plan}
% Describe your comprehensive plan for sharing research results.
%
% Extension Activities:
% - Field days, demonstrations, workshops for farmers/practitioners
% - Extension bulletins, fact sheets, online resources
% - Partnerships with Cooperative Extension Services
% - Industry engagement and technology transfer
%
% Education Activities:
% - Undergraduate/graduate student training
% - Curriculum development
% - K-12 outreach programs
% - Workshops and training sessions
%
% Scientific Communication:
% - Peer-reviewed publications (target journals)
% - Conference presentations
% - Webinars and podcasts
% - Open-access data and code repositories
%
% Public Communication:
% - Website and social media
% - Press releases and media engagement
% - Public seminars and community events
% - Policy briefs for decision-makers
%
% Evaluation:
% - How will you measure the reach and impact of your outreach?
% - What metrics will you track? (participants, downloads, citations, adoption)

%%%%%%%%%%%%%%%%%%%%%%%%%%%%%%%%%%%%%%%%%%%%%%%%%%%%%%%%%%%%%%%%%%%%%%%%%%%%%
% SECTION F: POTENTIAL LIMITATIONS AND PITFALLS
%%%%%%%%%%%%%%%%%%%%%%%%%%%%%%%%%%%%%%%%%%%%%%%%%%%%%%%%%%%%%%%%%%%%%%%%%%%%%
% Purpose: Demonstrate awareness of challenges and your preparedness to address them.
%
% What to include:
% - Anticipated technical or logistical challenges
% - Alternative approaches if methods fail
% - Risk mitigation strategies
% - Contingency plans
%%%%%%%%%%%%%%%%%%%%%%%%%%%%%%%%%%%%%%%%%%%%%%%%%%%%%%%%%%%%%%%%%%%%%%%%%%%%%

\section{Potential Limitations and Pitfalls}
% Identify potential challenges and your contingency plans.
%
% For each limitation, address:
% - What could go wrong or limit success?
% - Why might this occur?
% - What is the likelihood and potential impact?
% - What alternative approaches will you use?
% - How will you monitor and respond to these challenges?
%
% Common categories to consider:
% - Technical challenges (methodology, equipment, analysis)
% - Environmental factors (weather, pests, disease)
% - Logistical issues (recruitment, timing, resources)
% - Data quality or quantity concerns
% - Collaboration or stakeholder engagement challenges
%
% Tips:
% - Be honest but don't undermine confidence in your project
% - Show you've thought critically about potential problems
% - Demonstrate you have viable backup plans
% - Use the format: "If X occurs, we will do Y"

%%%%%%%%%%%%%%%%%%%%%%%%%%%%%%%%%%%%%%%%%%%%%%%%%%%%%%%%%%%%%%%%%%%%%%%%%%%%%
% SECTION G: SUSTAINABILITY PLAN
%%%%%%%%%%%%%%%%%%%%%%%%%%%%%%%%%%%%%%%%%%%%%%%%%%%%%%%%%%%%%%%%%%%%%%%%%%%%%
% Purpose: Explain how the project will continue after USDA funding ends.
%
% What to include:
% - Plans for continued funding (grant proposals, partnerships)
% - Long-term maintenance of products/resources
% - Institutionalization of programs or practices
% - Pathways to commercialization or adoption
%%%%%%%%%%%%%%%%%%%%%%%%%%%%%%%%%%%%%%%%%%%%%%%%%%%%%%%%%%%%%%%%%%%%%%%%%%%%%

\section{Sustainability Plan}
% Describe how project activities and impacts will be sustained beyond the funding period.
%
% Consider addressing:
%
% Financial Sustainability:
% - What funding sources will you pursue? (grants, partnerships, fees)
% - Any commercialization potential?
% - Self-sustaining revenue models?
%
% Operational Sustainability:
% - How will databases, websites, or tools be maintained?
% - Who will provide long-term support?
% - Institutional commitments?
%
% Impact Sustainability:
% - How will research findings continue to be used?
% - Pathways to practice adoption by stakeholders
% - Training programs that will continue
% - Ongoing partnerships and collaborations
%
% Scalability:
% - How can successful outcomes be expanded?
% - Potential for replication in other regions/systems
% - Transferability to related applications

%%%%%%%%%%%%%%%%%%%%%%%%%%%%%%%%%%%%%%%%%%%%%%%%%%%%%%%%%%%%%%%%%%%%%%%%%%%%%
% SECTION H: TIMELINE
%%%%%%%%%%%%%%%%%%%%%%%%%%%%%%%%%%%%%%%%%%%%%%%%%%%%%%%%%%%%%%%%%%%%%%%%%%%%%
% Purpose: Provide a realistic schedule showing when activities will occur.
%
% What to include:
% - Major activities organized by objective
% - Milestones and deliverables
% - Logical sequence and dependencies
% - Allocation across project years
%%%%%%%%%%%%%%%%%%%%%%%%%%%%%%%%%%%%%%%%%%%%%%%%%%%%%%%%%%%%%%%%%%%%%%%%%%%%%

\section{Timeline}
% Provide a detailed timeline of project activities.
%
% Recommended format:
% - Organize by objective or by year
% - List specific activities with start/end dates or year/quarter
% - Identify key milestones and decision points
% - Show dependencies between activities
%
% Consider using a table format:
% \begin{table}[h]
% \centering
% \begin{tabular}{|l|c|c|c|}
% \hline
% \textbf{Activity} & \textbf{Year 1} & \textbf{Year 2} & \textbf{Year 3} \\
% \hline
% Objective 1: Activity 1 & X & & \\
% Objective 1: Activity 2 & X & X & \\
% Objective 2: Activity 1 & & X & X \\
% \hline
% \end{tabular}
% \end{table}
%
% Or use a narrative format with subheadings:
% \boldline{Year 1 (Months 1-12)}
% - List activities and milestones
%
% \boldline{Year 2 (Months 13-24)}
% - List activities and milestones
%
% Include:
% - Field work seasons (planting, harvest)
% - Data collection periods
% - Analysis phases
% - Stakeholder engagement activities
% - Publication and dissemination activities
% - Project meetings and assessments

%%%%%%%%%%%%%%%%%%%%%%%%%%%%%%%%%%%%%%%%%%%%%%%%%%%%%%%%%%%%%%%%%%%%%%%%%%%%%
% REFERENCES / LITERATURE CITED
%%%%%%%%%%%%%%%%%%%%%%%%%%%%%%%%%%%%%%%%%%%%%%%%%%%%%%%%%%%%%%%%%%%%%%%%%%%%%
% IMPORTANT: USDA AFRI requires the bibliography to be submitted as a 
% SEPARATE PDF file. Do NOT include the bibliography in this narrative PDF.
%
% TO CREATE YOUR BIBLIOGRAPHY PDF:
% 1. Use the provided Bibliography.tex file
% 2. Compile it separately: lualatex Bibliography.tex (recommended)
% 3. Submit the resulting Bibliography.pdf as "Literature Cited"
%
% CITATION GUIDELINES:
% - Use \cite{key} throughout your narrative to cite references
% - All cited works will automatically appear in Bibliography.pdf
% - The bibliography does NOT count toward the 18-page narrative limit
% - Cite primary literature (avoid excessive review articles)
% - Include recent publications (last 5 years when possible)
% - Cite your own preliminary work when relevant
% - Use DOIs for journal articles
% - Ensure all citations are added to references.bib
%
% The bibliography section has been REMOVED from this file and moved to
% Bibliography.tex to comply with USDA AFRI submission requirements.
%%%%%%%%%%%%%%%%%%%%%%%%%%%%%%%%%%%%%%%%%%%%%%%%%%%%%%%%%%%%%%%%%%%%%%%%%%%%%

% DO NOT uncomment the bibliography lines below - use Bibliography.tex instead
% \bibliographystyle{unsrtnat}
% \bibliography{references}

\end{document}  