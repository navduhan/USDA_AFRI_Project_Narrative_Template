\documentclass{usda-afri-narrative}

%%%%%%%%%%%%%%%%%%%%%%%%%%%%%%%%%%%%%%%%%%%%%%%%%%%%%%%%%%%%%%%%%%%%%%%%%%%%%
% USDA AFRI Project Summary Template
% 
% IMPORTANT REQUIREMENTS:
% - Maximum 1 page
% - Must include: Project title, PD/Co-PDs/Co-PIs with institutions,
%   Program Area Priority code, and project summary paragraph
% - Use clear, concise language accessible to non-specialists
%
% COMPILATION:
% Recommended: lualatex ProjectSummary.tex (best Times New Roman font)
% Alternative: xelatex or pdflatex
%%%%%%%%%%%%%%%%%%%%%%%%%%%%%%%%%%%%%%%%%%%%%%%%%%%%%%%%%%%%%%%%%%%%%%%%%%%%%

\begin{document}

%%%%%%%%%%%%%%%%%%%%%%%%%%%%%%%%%%%%%%%%%%%%%%%%%%%%%%%%%%%%%%%%%%%%%%%%%%%%%
% PROJECT SUMMARY
%%%%%%%%%%%%%%%%%%%%%%%%%%%%%%%%%%%%%%%%%%%%%%%%%%%%%%%%%%%%%%%%%%%%%%%%%%%%%
% The Project Summary is a critical component of your USDA AFRI application.
% Reviewers often read this first, and program officers use it to assign
% reviewers and assess relevance to program priorities.
%
% KEY GUIDELINES:
% - Maximum 1 page
% - Write for a scientifically literate but non-specialist audience
% - Clearly state the problem, approach, and expected impact
% - Highlight relevance to USDA priorities and stakeholders
% - Avoid jargon and define any necessary technical terms
%%%%%%%%%%%%%%%%%%%%%%%%%%%%%%%%%%%%%%%%%%%%%%%%%%%%%%%%%%%%%%%%%%%%%%%%%%%%%

\begin{center}
\textbf{\large Project Summary}
\end{center}

\parspace

%%%%%%%%%%%%%%%%%%%%%%%%%%%%%%%%%%%%%%%%%%%%%%%%%%%%%%%%%%%%%%%%%%%%%%%%%%%%%
% PROJECT TITLE
%%%%%%%%%%%%%%%%%%%%%%%%%%%%%%%%%%%%%%%%%%%%%%%%%%%%%%%%%%%%%%%%%%%%%%%%%%%%%
% Use a clear, descriptive title that:
% - Captures the essence of your project
% - Is accessible to non-specialists
% - Includes key concepts or species/systems
% - Matches the title in your application forms
%%%%%%%%%%%%%%%%%%%%%%%%%%%%%%%%%%%%%%%%%%%%%%%%%%%%%%%%%%%%%%%%%%%%%%%%%%%%%

\boldline{Project title:}
[Your Full Project Title Here - Should Match Application Forms]

\parspace

%%%%%%%%%%%%%%%%%%%%%%%%%%%%%%%%%%%%%%%%%%%%%%%%%%%%%%%%%%%%%%%%%%%%%%%%%%%%%
% PRINCIPAL INVESTIGATORS AND INSTITUTIONS
%%%%%%%%%%%%%%%%%%%%%%%%%%%%%%%%%%%%%%%%%%%%%%%%%%%%%%%%%%%%%%%%%%%%%%%%%%%%%
% List PD (Project Director), Co-PDs, and Co-PIs with their institutions.
% Format: One line per person or use a table for multiple investigators.
%%%%%%%%%%%%%%%%%%%%%%%%%%%%%%%%%%%%%%%%%%%%%%%%%%%%%%%%%%%%%%%%%%%%%%%%%%%%%

\noindent
\begin{tabular}{@{}p{0.48\textwidth}@{\hspace{0.04\textwidth}}p{0.48\textwidth}@{}}
\textbf{PD:} [Last Name, First Name] & \textbf{Institution:} [Your Institution Name] \\
\textbf{Co-PD:} [Last Name, First Name] & \textbf{Institution:} [Institution Name] \\
\textbf{Co-PI:} [Last Name, First Name] & \textbf{Institution:} [Institution Name] \\
\end{tabular}

% For single PD, use this simpler format instead:
% \textbf{PD:} [Last Name, First Name] \hfill \textbf{Institution:} [Your Institution Name]

\parspace

%%%%%%%%%%%%%%%%%%%%%%%%%%%%%%%%%%%%%%%%%%%%%%%%%%%%%%%%%%%%%%%%%%%%%%%%%%%%%
% PROGRAM AREA PRIORITY CODE
%%%%%%%%%%%%%%%%%%%%%%%%%%%%%%%%%%%%%%%%%%%%%%%%%%%%%%%%%%%%%%%%%%%%%%%%%%%%%
% Specify the USDA AFRI Program Area Priority code from the RFA.
% Example codes: A1101, A1701, A1431, etc.
% This must match what you select in your application forms.
%%%%%%%%%%%%%%%%%%%%%%%%%%%%%%%%%%%%%%%%%%%%%%%%%%%%%%%%%%%%%%%%%%%%%%%%%%%%%

\boldline{Program Area Priority (Program Code):}
[A\#\#\#\# - Brief Program Name from RFA]

\parspace

%%%%%%%%%%%%%%%%%%%%%%%%%%%%%%%%%%%%%%%%%%%%%%%%%%%%%%%%%%%%%%%%%%%%%%%%%%%%%
% PROJECT SUMMARY PARAGRAPH
%%%%%%%%%%%%%%%%%%%%%%%%%%%%%%%%%%%%%%%%%%%%%%%%%%%%%%%%%%%%%%%%%%%%%%%%%%%%%
% This is the heart of your Project Summary. Write a compelling narrative
% (typically 300-500 words) that addresses:
%
% 1. PROBLEM/NEED (2-3 sentences)
%    - What is the critical agricultural challenge?
%    - Why is it important? (economic, food security, sustainability)
%    - What are current limitations or knowledge gaps?
%
% 2. LONG-TERM GOAL (1 sentence)
%    - What is the overarching aim of your research program?
%
% 3. OBJECTIVES (3-5 sentences)
%    - What specific aims will you pursue?
%    - List 3-5 numbered objectives
%    - Keep objectives clear and measurable
%
% 4. HYPOTHESIS (1-2 sentences)
%    - What is your central hypothesis or rationale?
%    - What do you expect to discover/demonstrate?
%
% 5. RELEVANCE TO PROGRAM AREA (1-2 sentences)
%    - How does this align with USDA AFRI priorities?
%    - Explicitly mention the program area
%
% 6. EXPECTED OUTCOMES/IMPACT (2-3 sentences)
%    - What will be the tangible results?
%    - Who will benefit? (farmers, industry, consumers, environment)
%    - How will this address industry/stakeholder needs?
%
% WRITING TIPS:
% - Use active voice
% - Avoid excessive jargon
% - Define technical terms when first used
% - Emphasize practical applications and impacts
% - Quantify when possible (e.g., $X billion industry, Y% losses)
% - Connect to USDA strategic goals
%%%%%%%%%%%%%%%%%%%%%%%%%%%%%%%%%%%%%%%%%%%%%%%%%%%%%%%%%%%%%%%%%%%%%%%%%%%%%

\boldline{Project Summary:}

[Begin with 2-3 sentences describing the problem/need. Explain the agricultural challenge, its importance, and current limitations.]

[State your long-term goal in one clear sentence.]

[Present your specific objectives. Consider this format:]
Specific objectives are to: 
1) [first objective using an action verb]; 
2) [second objective]; 
3) [third objective]; 
4) [fourth objective, if applicable].

[State your hypothesis or central rationale. What do you expect to discover or demonstrate?]

[Explain how this project aligns with the USDA AFRI program area. Explicitly mention the program code and priority area.]

[Describe expected outcomes and impacts. Who will benefit and how? What tangible products or knowledge will be delivered? How does this address stakeholder needs?]



\end{document}
